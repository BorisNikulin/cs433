\documentclass{article}

\usepackage{microtype}
\usepackage{parskip}
\usepackage[usenames, dvipsnames]{xcolor}

\usepackage{hyperref}
\hypersetup{
	colorlinks=true,
	linkcolor=blue,
	filecolor=magenta,
	urlcolor=cyan!80!black,
	citecolor=orange!80!black,
}


\title{CS 433 (2) HW4 Report}
\date{2019-04-22}
\author{Boris Nikulin}

\begin{document}

\maketitle
\tableofcontents

\section{Solving the Producer Consumer Problem}

To solve the producer consumer problem,
I used a mutex and two condition variables.

The buffer being produced and consumed to
is uninitialized raw storage represented.


\subsection{The Buffer}

The buffer was made with uninitialized raw storage to
prevent unnecessary initialization.

The buffer is created using an array of bytes aligned to the element type.
The buffer was handled manually as an exercise instead of using \texttt{std::aligned\_storage}.

The buffer is used in a FIFO manner.


\subsection{Synchronization}

Synchronization is primarily attained through a single mutex.
The single mutex is responsible for synchronizing access to the buffer.

When the buffer is empty or full,
threads wait on a non empty and a not full condition variables respectively.
The conditions variables are signalled when
an element has been popped off the FIFO buffer or
when an element has been added.
These two condition variables block when an action can not be possible,
while the opposite action signals the condition variable and enabling the action.


\end{document}
